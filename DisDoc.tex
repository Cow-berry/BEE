\documentclass[12pt,a4paper, titlepage]{article}
\usepackage[utf8]{inputenc}
\usepackage[russian]{babel}
\usepackage[OT1]{fontenc}
\usepackage{amsmath}
\usepackage{amsfonts}
\usepackage{amssymb}
\usepackage{graphicx}
\usepackage[left=1cm,right=1cm,top=2cm,bottom=2cm]{geometry}
\title{BEE Project}
\author{Разработчики: \\Каданцев Георгий \& Коченюк Анатолий}
\begin{document}
\maketitle
\section{Обзор}

В первую очередь этот проект представляет собой симулятор эволюции. 

Пользователю будет предоставлена карта и некоторое число разных живых существ. Существа будут двигаться с помощью нейронной сети и размножаться согласно основам генетики. Также будет существовать пищевая цепочка(и).

В дополнение к обзору животных пользователю будет предоставлена некоторая статистика (графики роста популяции/ореолов обитания/преобладающие признаки у каждого вида в настоящий момент). Будет предоставлена возможность ускорять и замедлять все процессы. 

Также должно быть два меню: одно сверху (файл, настройки приложения, открыть/закрыть и т.д.). другое справа -- переключение между вкладками с разными графиками/картами/настройками процессов. 
\section{Итерации}
В этом разделе необходимо помечать даты выполнения каждой итерации
\begin{enumerate}
	\item 
	\begin{itemize}
		\item Базовый Интерфейс: Меню сверху, меню сбоку./подразумевает освоение с модулем PyQt5
		\item Нейронные сети для передвижения тел. 
	\end{itemize}
	\hrule
	\item 
	\begin{itemize}
		\item Полный интерфейс со всеми запланированными вкладками и функционалом (настоящим или потенциальным)
		\item Алгоритмы Рекомбинации + Изменчивость
	\end{itemize}
	\hrule
	\item 
	\begin{itemize}
		\item Проектирование баз данных для сохранения информации о карте и телах.
		\item Генерация Ландшафта
	\end{itemize}
\end{enumerate}
\section{Предложенное решение (Техническая архитектура)}
Со временем нужно распланировать каждый из следующих подпунктов, чтобы всегда знать, что уже сделано и что предстоит делать дальше.
\subsection{Интерфейс}
Окно будет содержать обычное меню сверху, а также переключение вкладок справа. Изначально будет показываться вкладка с картой, но также будет карта со статистикой, управлением...

На данный момент представлен файлом ({\tt app/window.py}) и служит для теста нейронки.
\subsection{Генерация Ландшафта}
\subsection{Базы данных}
\subsection{Нейронные сети}
Небольшая нейронная сеть ({\tt app/organism.py}). По относительному расположению еды, а также своему положению определяет куда идти.

Её необходимо натренировать хотя бы идти на еду.
\subsection{Генетика (Наследственность и Изменчивость)}
Каждый организм -- отдельная маленькая нейронка со своими весами.
При образовании потомства, берутся средние арифметические весов нейронок и прибавляются некоторые мутации.

Потом появятся гены и работа с ними.
\end{document}
