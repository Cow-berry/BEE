\documentclass[12pt,a4paper, titlepage]{article}
\usepackage[utf8]{inputenc}
\usepackage[russian]{babel}
\usepackage[OT1]{fontenc}
\usepackage{amsmath}
\usepackage{amsfonts}
\usepackage{amssymb}
\usepackage{graphicx}
\usepackage[left=1cm,right=1cm,top=2cm,bottom=2cm]{geometry}
\title{BEE Project}
\author{Разработчики: \\Каданцев Георгий \& Коченюк Анатолий}
\begin{document}
\maketitle
\section{Обзор}

Этот проект представляет собой эмулятор эволюции. 

Пользователю будет предоставлена карта и некоторое число разных существ.
Существа будут жить, размножаться согласно основам генетики. 
Поведение определяется простой нейронной сетью.
Также будет существовать пищевая цепочка(и).

В дополнение к обзору организмов пользователю будет предоставлена некоторая статистика (графики роста популяции/ореолов обитания/преобладающие признаки у каждого вида в настоящий момент). 
Будет предоставлена возможность ускорять и замедлять все процессы. 

Также должно быть два меню: одно сверху (файл, настройки приложения, открыть/закрыть и т.д.), другое справа -- переключение между вкладками с разными графиками/картами/настройками процессов. 
\section{Итерации}
В этом разделе необходимо помечать даты выполнения каждой итерации
\begin{enumerate}
	\item 
	\begin{itemize}
		\item Базовый Интерфейс: меню сверху, меню сбоку./подразумевает освоение с модулем PyQt5 --- \textbf{выполнено, 28.11}
		\item Нейронные сети для передвижения тел.--- \textbf{выполнено, 28.11} 
	\end{itemize}
	\hrule
	\item 
	\begin{itemize}
		\item Проектирование баз данных для сохранения информации о карте и телах.
		\item Алгоритмы Рекомбинации + Изменчивость
	\end{itemize}
	\hrule
	\item 
	\begin{itemize}
		\item Полный интерфейс со всеми запланированными вкладками и функционалом (настоящим или потенциальным)
		\item Генерация Ландшафта
	\end{itemize}
\end{enumerate}
\section{Предложенное решение (Техническая архитектура)}
На данный момент разрабатывается простой эволюционный алгоритм обучения нейронной сети задаче подойти к ресурсу.
Алгоритм:
	\begin{enumerate}
		\item \textbf{Инициализация}. 
		Создается первое поколение/популяция.
		Характеристики особей (веса нейронки, цвет) определяются случайным образом.
		\item \textbf{Пробы}.
		Особям предоставляется определенное количество шагов, по истечению которых определяется результат особи (функция приспособления --- расстояние до ресурса)
		\item \textbf{Отбор}.
		По результатам "забега" особи ранжируются и как-то определяется набор организмов, прошедших отбор.
		Они принимают участие в следующем этапе, остальные особи "умирают".
		\item \textbf{Размножение и мутирование}.
		Набор особей разбивается на пары, которые скрещиваются --- берутся их параметры, усредняются и добавляются случайные \textit{мутации}.
		Получаются наборы новых организмов (от одних родителей), которые потом все вместе принимают участие в пробах и цикл продолжается оттуда.  
\end{enumerate}
		
На данный момент отбор запрограммирован так: особи сортируются по функции приспособления и проходят отбор особи лучшей половины результатов. 
Этот подход признан неэффективным.

Задача: реализовать такой отбор: особь проходит отбор с шансом $1 - \frac{n}{A}$, где $n$ -- номер в рейтинге, $A$ -- общее количество организмов в каждой популяции. 
	  
\subsection{Интерфейс}
Окно будет содержать обычное меню сверху, а также переключение вкладок справа. Изначально будет показываться вкладка с картой, но также будет карта со статистикой, управлением...

На данный момент представлен файлом ({\tt app/window.py}).

\subsection{Базы данных}
Для сбора данных и вывода разных графиков будут использоваться таблицы csv, которые будут обрабатываться с помощью модуля {\tt pandas}.
\subsection{Нейронные сети}
Небольшая нейронная сеть ({\tt app/organism.py}). 
По относительному расположению еды ресурсов определяет куда идти.
Реализуется с помощью модуля {\tt pybrain}.
\subsection{Генетика (Наследственность и Изменчивость)}
Каждый организм -- отдельная маленькая нейронка со своими весами.
При образовании потомства, берутся средние арифметические весов нейронок и прибавляются некоторые мутации.

Гены и работа с ними должны будут быть в {\tt app/organism.py}.

\subsection{Генерация Ландшафта}
\end{document}
